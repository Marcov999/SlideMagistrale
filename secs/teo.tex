\begin{frame}
  \frametitle{Un teorema di tipo ``Wolff-Denjoy'' per sottovarietà relativamente compatte}
  \only<1-2>{
    \begin{defn}
      Una varietà complessa $X$ si dice \textit{taut} se ogni funzione nella chiusura (rispetto alla topologia compatta-aperta) di $\text{Hol}(\mathbb{D},X)$ in $C^0(\mathbb{D},X^*)$ è in $\text{Hol}(\mathbb{D},X)$ oppure è la funzione costante $\infty$.
  \end{defn}
  }
  \only<2>{
    Si può dimostrare che ogni varietà taut è Kobayashi-iperbolica.
  }
  \only<3->{
    \begin{block}{Teorema (Chandel, Maitra, Sarkar, 2021; Bharali, Zimmer, 2022)}\begin{itshape}
      Sia $X$ una sottovarietà taut e relativamente compatta di una varietà complessa $Y$. \setcounter{beamerpauses}{3}\pause Supponiamo che esista un $\kappa_0>0$ tale che $X$ sia $(1,\kappa_0)$-visibile.
    
      Sia $F:X \longrightarrow X$ una funzione olomorfa. \pause Allora vale esattamente una delle seguenti affermazioni: \pause
      \begin{itemize}
        \item le orbite dei punti di $X$ tramite $F$ sono relativamente compatte in $X$; \pause oppure,
        \item esiste un unico punto di $\partial_YX$ tale che la successione delle iterate di $F$ converge, uniformemente sui compatti, a quel punto.
      \end{itemize}
    \end{itshape}\end{block}
  }
\end{frame}

\begin{frame}
  \frametitle{Strada per la dimostrazione del teorema di BB}
  \begin{enumerate}
    \item Dato $(Z,d)$ spazio metrico completo e limitato, si può costruire uno spazio Gromov-iperbolico $\big(\text{Con}(Z),r\big)$ tale che $Z$ è identificato con il bordo.
    \pause
    \item La metrica di Kobayashi soddisfa una particolare catena di disuguaglianze; \pause usando tali disuguaglianze si può dimostrare che $k_{\Omega}$, vicino al bordo, differisce per una costante da una certa funzione $g$ che è sostanzialmente l'equivalente di $r$ per $\Omega$ (si dice che $k_{\Omega}$ e $g$ sono quasi-isometriche).
    \pause
    \item Poiché la Gromov-iperbolicità è invariante per quasi-isometrie, questo ci permette di dire che $(\Omega,k_{\Omega})$ è Gromov-iperbolico.
  \end{enumerate}
\end{frame}
