\begin{frame}
  \frametitle{Un teorema di tipo ``Wolff-Denjoy'' per sottovarietà relativamente compatte}
  \only<1-2>{
    \begin{defn}
      Una varietà complessa $X$ si dice \textit{taut} se ogni funzione nella chiusura (rispetto alla topologia compatta-aperta) di $\text{Hol}(\mathbb{D},X)$ in $C^0(\mathbb{D},X^*)$ è in $\text{Hol}(\mathbb{D},X)$ oppure è la funzione costante $\infty$.
  \end{defn}\pause

    Si può dimostrare che ogni varietà taut è Kobayashi-iperbolica.
  }
  \only<3->{
    \begin{block}{Teorema (Chandel, Maitra, Sarkar, 2021; Bharali, Zimmer, 2022)}\begin{itshape}
      Sia $X$ una sottovarietà taut e relativamente compatta di una varietà complessa $Y$. \setcounter{beamerpauses}{3}\pause Supponiamo che esista un $\kappa_0>0$ tale che $X$ sia $(1,\kappa_0)$-visibile.
    
      Sia $F:X \longrightarrow X$ una funzione olomorfa. \pause Allora vale esattamente una delle seguenti affermazioni: \pause
      \begin{itemize}
        \item le orbite dei punti di $X$ tramite $F$ sono relativamente compatte in $X$; \pause oppure,
        \item esiste un unico punto di $\partial_YX$ tale che la successione delle iterate di $F$ converge, uniformemente sui compatti, a quel punto.
      \end{itemize}
    \end{itshape}\end{block}
  }
\end{frame}

\begin{frame}
  \frametitle{Strada per la dimostrazione del teorema di tipo ``Wolff-Denjoy''}
  \begin{enumerate}
    \item dall'ipotesi che la varietà sia taut, per un teorema di Abate segue che se le orbite non sono relativamente compatte allora la successione delle iterate è compattamente divergente;\pause
    \item dalle ipotesi di visibilità e di relativa compattezza segue, a meno di sottosuccessioni, la convergenza uniforme sui compatti a una costante nel bordo della varietà;\pause
    \item sempre per la condizione di visibilità, tale limite è lo stesso per ogni sottosuccessione, dunque dev'essere il limite di tutta la successione.
  \end{enumerate}
\end{frame}