\documentclass{beamer}
\usepackage{mstyle}
\usepackage{pgfplots}
\usepackage{appendixnumberbeamer}
\usetheme{Madrid}
\usefonttheme{serif}
\usetikzlibrary{intersections, pgfplots.fillbetween}
\setbeamertemplate{enumerate items}[default]
\makeatother
\setbeamertemplate{footline}
{
  \leavevmode%
  \hbox{%
  \begin{beamercolorbox}[wd=.2\paperwidth,ht=2.25ex,dp=1ex,center]{author in head/foot}%
    \usebeamerfont{author in head/foot}{Marco Vergamini}
  \end{beamercolorbox}%
  \begin{beamercolorbox}[wd=.55\paperwidth,ht=2.25ex,dp=1ex,center]{title in head/foot}%
    \usebeamerfont{title in head/foot}\insertshorttitle%\hspace*{3em}
  \end{beamercolorbox}%
  \begin{beamercolorbox}[wd=.25\paperwidth,ht=2.25ex,dp=1ex,center]{date in head/foot}%
    \usebeamerfont{date in head/foot}\insertshortdate\hspace*{1em}
    \insertframenumber{}/\inserttotalframenumber
  \end{beamercolorbox}}%
  \vskip0pt%
}
\makeatletter
\setbeamertemplate{navigation symbols}{}

\title{Teoremi di tipo ``Wolff-Denjoy'' in più variabili complesse}
\institute{Università di Pisa \\
Corso di Laurea Magistrale in Matematica}
\date{22 Settembre 2023}
\titlegraphic{\includegraphics[width=0.30\textwidth]{Stemma_unipi.jpg}}
\author{Candidato: Marco Vergamini \qquad Relatore: Prof. Marco Abate}

\begin{document}

\begin{frame}
  \begin{beamercolorbox}[sep=8pt,center,colsep=-4bp,rounded=true,shadow=true]{title}
    \usebeamerfont{title}\inserttitle\par%
    \ifx\insertsubtitle\@empty%
    \else%
    \vskip0.25em%
    {\usebeamerfont{subtitle}\usebeamercolor[fg]{subtitle}\insertsubtitle\par}%
    \fi%
  \end{beamercolorbox}

  \begin{beamercolorbox}[sep=8pt,center,colsep=-4bp,rounded=true,shadow=true]{date}
    \usebeamerfont{date}\insertdate
  \end{beamercolorbox}\vskip0.5em

  \begin{center}
    {\usebeamercolor[fg]{titlegraphic}\inserttitlegraphic\par}
  \end{center}

  \begin{beamercolorbox}[sep=8pt,center,colsep=-4bp,rounded=true,shadow=true]{institute}
    \usebeamerfont{institute}\insertinstitute
  \end{beamercolorbox}

  \begin{beamercolorbox}[sep=8pt,center,colsep=-4bp,rounded=true,shadow=true]{author}
    \usebeamerfont{author}\insertauthor
  \end{beamercolorbox}
\end{frame}

\begin{frame}
  \frametitle{Il teorema di Wolff-Denjoy}
  Sia $\mathbb{D}$ il disco unitario in $\mathbb{C}$.\pause
  \begin{block}{Teorema (Wolff-Denjoy, 1926)}
    \begin{itshape}
      Sia $f:\mathbb{D} \longrightarrow \mathbb{D}$ una funzione olomorfa. \pause Allora vale esattamente una delle seguenti affermazioni:\pause
      \begin{itemize}
        \item la funzione $f$ ha un punto fisso nel disco; \pause oppure,
        \item esiste un unico punto del bordo del disco tale che la successione delle iterate di $f$ converge, uniformemente sui compatti, a quel punto.
      \end{itemize}
    \end{itshape}
  \end{block}
\end{frame}

\begin{frame}[t]
  \frametitle{Generalizzazione ai domini limitati e strettamente pseudoconvessi}
  \only<1>{
    \begin{defn}
      La \textit{distanza di Poincaré} (o \textit{iperbolica}) $\omega$ su $\mathbb{D}$ è data da
      \begin{equation*}
          \omega(z_1,z_2)=\frac{1}{2}\log{\frac{1+\left|\frac{z_1-z_2}{1-\bar{z}_1z_2}\right|}{1-\left|\frac{z_1-z_2}{1-\bar{z}_1z_2}\right|}}
      \end{equation*}
      per ogni $z_1,z_2 \in \mathbb{D}$.
  \end{defn}
  }
  \only<2-3>{
    \begin{defn}
      Sia $X$ una varietà complessa e connessa; la \textit{pseudodistanza di Kobayashi} su $X$ è data da
    \begin{equation*}\begin{split}
        k_X(z,w)=&\inf\Bigg\{\sum_{j=1}^m \omega(\zeta_{j-1},\zeta_j) \bigg\vert \text{esistono }m\in\mathbb{N},\text{ punti }\zeta_0,\dots,\zeta_m \in \mathbb{D}\text{ e}\\
        &\text{funzioni }\varphi_1,\dots,\varphi_m\in\text{Hol}(\mathbb{D},X) \text{ tali che } \varphi_1(\zeta_0)=z,\\
        &\varphi_m(\zeta_m)=w\text{ e }\varphi_j(\zeta_j)=\varphi_{j+1}(\zeta_j)\text{ per }j=1,\dots,m-1\Bigg\}
    \end{split}\end{equation*}
    per $z,w \in X$.
    
    \only<3>{Se $k_X$ è una distanza, diremo che $X$ è \textit{Kobayashi-iperbolica}.}
    \end{defn}
  }
  \only<4->{
  \begin{block}{Teorema (Abate, 1991)}\begin{itshape}
    Siano $\Omega\subseteq\mathbb{C}^n$ un dominio limitato e strettamente pseudoconvesso e $f:\Omega \longrightarrow \Omega$ una funzione olomorfa. \setcounter{beamerpauses}{4}\pause Allora vale esattamente una delle seguenti affermazioni: \pause
    \begin{enumerate}
      \item le orbite di $f$ sono relativamente compatte in $\Omega$; \pause oppure,
      \item esiste un unico punto di $\partial\Omega$ tale che le iterate di $f$ convergono tutte, uniformemente sui compatti, a quel punto.
    \end{enumerate}
  \end{itshape}\end{block}}\only<8->{
    \textit{Traccia della dimostrazione:} Per un teorema di Balogh e Bonk del 2000, $(\Omega,k_\Omega)$ è uno spazio metrico Gromov-iperbolico. \only<9->{

    Allora soddisfa le ipotesi di un teorema di Karlsson del 2001, per cui le orbite sono limitate (in $k_\Omega$) o convergono a un unico punto del bordo.} \only<10->{

    Per avere la convergenza uniforme sui compatti si applica il teorema di Montel. \qed}
  }
\end{frame}

\begin{frame}[t]
  \frametitle{La condizione di visibilità}
  \only<1>{
    \begin{defn}
      Sia $X$ una varietà complessa; la \textit{pseudometrica di Kobayashi} su $X$ è
      \begin{equation*}\begin{split}
          K_X(x;Z)=&\inf\{|v| \mid v \in \mathbb{C}, \text{ esiste }f \in \text{Hol}(\mathbb{D},X) \\
          &\text{ tale che } f(0)=x, \diff_0 f(v)=Z\}
      \end{split}\end{equation*}
      per ogni $x \in X$ e $Z \in T_xX$.
    \end{defn}
  }
  \only<2-5>{
    \begin{defn}
      Sia $X$ una varietà complessa e connessa; fissiamo due costanti $\lambda \ge 1$ e $\kappa \ge 0$. Sia $I\subseteq \mathbb{R}$ un intervallo; \setcounter{beamerpauses}{2}\pause una curva $\sigma:I \longrightarrow X$ è detta una \textit{$(\lambda,\kappa)$-simil-geodetica} se:\pause
    \begin{enumerate}
        \item per ogni $s,t \in I$ si ha
        \begin{equation*}
            \frac{1}{\lambda}|t-s|-\kappa \le k_X\big(\sigma(s),\sigma(t)\big)\le\lambda|t-s|+\kappa;
        \end{equation*}\pause
        \item $\sigma$ è assolutamente continua rispetto a $d_X$ (quindi $\sigma'(t)$ esiste per quasi ogni $t \in I$) e per quasi ogni $t \in I$ si ha
        \begin{equation*}
            K_X\big(\sigma(t);\sigma'(t)\big) \le \lambda.
        \end{equation*}
    \end{enumerate}
    \end{defn}
  }
  \only<6-9>{
    \begin{defn}
      Sia $X$ una sottovarietà complessa e connessa di una varietà complessa $Y$, e fissiamo $\lambda \ge 1$ e $\kappa \ge 0$. \setcounter{beamerpauses}{6}\pause Diciamo che $X$ è \textit{$(\lambda,\kappa)$-visibile} se:\pause
      \begin{enumerate}
          \item ogni due punti distinti di $X$ possono essere collegati da una $(\lambda,\kappa)$-simil-geodetica;\pause
          \item per ogni coppia di punti $p,q\in\partial_YX$ con $p\not=q$, esistono in $\overline{X}$ due intorni $V$ e $W$, di $p$ e $q$ rispettivamente, con chiusura disgiunta, e un compatto $K$ di $X$ tali che  ogni $(\lambda,\kappa)$-simil-geodetica in $X$ che collega un punto di $V$ a un punto di $W$ interseca $K$.
      \end{enumerate}
  \end{defn}
  }
  \only<10-14>{
    Caso escluso: le simil-geodetiche da $U$ a $V$ fuggono dal compatto $K$.
  }
  \only<10>{
    \includegraphics[width=1.05\textwidth, trim=0 18cm 0 3cm]{vis1.png}
  }
  \only<11>{
    \includegraphics[width=1.05\textwidth, trim=0 18cm 0 3cm]{vis2.png}
  }
  \only<12>{
    \includegraphics[width=1.05\textwidth, trim=0 18cm 0 3cm]{vis3.png}
  }
  \only<13>{
    \includegraphics[width=1.05\textwidth, trim=0 18cm 0 3cm]{nonvis4.png}
  }
  \only<14>{
    \includegraphics[width=1.05\textwidth, trim=0 18cm 0 3cm]{nonvis5.png}
  }
  \only<15>{
    \includegraphics[width=1.05\textwidth, trim=0 16cm 0 3cm]{vis1.png}
    Condizione di visibilità: le simil-geodetiche ``curvano verso l'interno'', rimanendo dentro il compatto $K$.
  }
\end{frame}

\begin{frame}
  \frametitle{Strada per la dimostrazione del teorema di BB}
  \begin{enumerate}
    \item Dato $(Z,d)$ spazio metrico completo e limitato, si può costruire uno spazio Gromov-iperbolico $\big(\text{Con}(Z),r\big)$ tale che $Z$ è identificato con il bordo.
    \pause
    \item La metrica di Kobayashi soddisfa una particolare catena di disuguaglianze; \pause usando tali disuguaglianze si può dimostrare che $k_{\Omega}$, vicino al bordo, differisce per una costante da una certa funzione $g$ che è sostanzialmente l'equivalente di $r$ per $\Omega$ (si dice che $k_{\Omega}$ e $g$ sono quasi-isometriche).
    \pause
    \item Poiché la Gromov-iperbolicità è invariante per quasi-isometrie, questo ci permette di dire che $(\Omega,k_{\Omega})$ è Gromov-iperbolico.
  \end{enumerate}
\end{frame}

\begin{frame}[t]
  \frametitle{1. Lo spazio iperbolico $\text{Con}(Z)$}
  \only<1-4>{
  \begin{block}{Teorema (Bonk-Schramm, 2000)}\begin{itshape}
    Sia $(Z,d)$ uno spazio metrico completo e limitato, e sia $\text{Con}(Z)=Z\times(0, D(Z)]$, dove $D(Z)$ è il diametro di $Z$.
    La funzione $r:\text{Con}(Z)\times\text{Con}(Z) \longrightarrow [0,+\infty)$ data da
    $$r\big((z,h),(z',h')\big)=2\log\left(\frac{d(z,z')+\max\{h, h'\}}{\sqrt{hh'}}\right)$$
    è una distanza su $\text{Con}(Z)$ che lo rende uno spazio Gromov-iperbolico, il cui bordo può essere identificato con $Z$.
  \end{itshape}\end{block}
  }
  \only<2-4>{
  \textit{Traccia della dimostrazione:} è facile verificare che $r$ è una distanza.

  }
  \only<3-4>{Dati $r_{ij} \ge 0$ per $i,j \in \{1,2,3,4\}$ tali che $r_{ij}=r_{ji}$ e $r_{ij} \le r_{ik}+r_{kj}$, allora $r_{12}r_{34} \le 4\cdot\max\{r_{13}r_{24}, r_{14}r_{23}\}$.

  }
  \only<4>{
  Poniamo $x_i=(z_i,h_i) \in \text{Con}(Z)$ per $i \in \{1,2,3,4\}$, $d_{ij}=d(z_i,z_j)$ e $r_{ij}=d_{ij}+\max\{h_i, h_j\}$.
  }
  \only<5-6>{Segue che
  \begin{align*}
    (d_{12}+\max\{h_1, h_2\}&)(d_{34}+\max\{h_3, h_4\}) \\
    &\le 4\max\{(d_{13}+\max\{h_1, h_3\})(d_{24}+\max\{h_2, h_4\}), \\
    &(d_{14}+\max\{h_1, h_4\})(d_{23}+\max\{h_2, h_3\})\},
  \end{align*}}
  \only<6>{che ci dà
  \begin{align*}
    r(x_1,x_2)&+r(x_3,x_4) \\
    & \le \max\{r(x_1,x_3)+r(x_2,x_4), r(x_1,x_4)+r(x_2,x_3)\}+C,
  \end{align*}
  da cui segue la Gromov-iperbolicità di $\big(\textit{Con}(Z), r\big)$.}
  \only<7->{
  Fissiamo $w=\big(z_0,D(Z)\big) \in \text{Con}(Z)$; usando le definizioni, troviamo che dati $x=(z,h),x'=(z',h')\in \text{Con}(Z)$ vale
  $$(x,x')_w=-\log\big(d(z,z')+\max\{h, h'\}\big)+O_{D(Z)}(1).$$}
  \only<8->{Segue che una successione $(x_i)$ in $\big(\text{Con}(Z),r\big)$ converge a infinito se e solo se la successione $(z_i)$ è di Cauchy e $h_i \longrightarrow 0$; essendo $Z$ completo, possiamo quindi associare a $(x_i)$, come ``limite'', un unico $z \in Z$.}
  \only<9->{
  Inoltre, due successioni convergenti a infinito sono equivalenti se e solo se il loro limite è lo stesso, e ogni punto di $Z$ è limite di una successione che converge a infinito; questo ci dà un'identificazione, come insiemi, di $Z$ e $\partial_G\text{Con}(Z)$.

  }
  \only<10->{
  Mettendo su $\text{Con}(Z)\cup\partial_G\text{Con}(Z)$ un'opportuna topologia (di compattificazione), segue anche che $Z$ e $\partial_G\text{Con}(Z)$ sono identificati come spazi topologici; \only<11>{questo non è difficile, ma richiede un po' di passaggi che non vedremo. \qed}
  }
\end{frame}

\begin{frame}[t]
  \frametitle{Intorno tubolare e tangente ortogonale complesso}
  \only<1->{
  Sia $N_{\epsilon}(\partial\Omega)=\{x \in \mathbb{C}^n \mid \delta(x)<\epsilon\}$. Si può dimostrare che esiste $\epsilon>0$ tale che $N_{\epsilon}(\partial\Omega)$ è un intorno tubolare di $\partial\Omega$.
  }
  \only<3->{
  Sia $\pi$ la proiezione su $\partial\Omega$; dati $x,y \in \Omega\cap N_{\epsilon}(\partial\Omega)$, poniamo
  $$g(x,y)=2\log\left(\frac{d_H\big(\pi(x),\pi(y)\big)+\max\{\delta(x)^{1/2 },\,\delta(y)^{1/2}\}}{\sqrt{\delta(x)^{1/2 }\delta(y)^{1/2}}}\right).$$
  }
  \only<4>{
  Fissato $p \in \partial\Omega$ e detta $\nu(p)$ la normale reale uscente da $\partial\Omega$ in $p$, possiamo decomporre $\mathbb{C}^n=H_p\partial\Omega\oplus \text{Span}_{\mathbb{C}}\{\nu(p)\}$;
  dato $Z \in \mathbb{C}^n$, scriviamo in modo unico $Z=Z_H+Z_N$ con $Z_H \in H_p\partial\Omega$ e $Z_N \in \text{Span}_{\mathbb{C}}\{\nu(p)\}$.
  }
\end{frame}

\begin{frame}[t]
  \frametitle{2. Stime per la metrica di Kobayashi}
  \only<1->{\begin{prop}
    Per ogni $\epsilon>0$ esistono $\epsilon_0>0$ e $C \ge 0$ tali che per ogni $x \in \Omega\cap N_{\epsilon_0}(\partial\Omega)$ e per ogni $Z \in \mathbb{C}^n$ si ha
    \begin{multline*}
      \big(1-C\delta^{1/2}(x)\big)\left(\frac{|Z_N|^2}{4\delta^2(x)}+(1-\epsilon)\frac{L_\rho\big(\pi(x);Z_H\big)}{\delta(x)}\right)^{1/2} \le K_{\Omega}(x;Z) \\
      \le \big(1+C\delta^{1/2}(x)\big)\left(\frac{|Z_N|^2}{4\delta^2(x)}+(1+\epsilon)\frac{L_\rho\big(\pi(x);Z_H\big)}{\delta(x)}\right)^{1/2}.
    \end{multline*}
  \end{prop}}
  \only<2-4>{\textit{Traccia della dimostrazione:} si localizza a un intorno di un punto del bordo.}\only<3-4>{ Con un opportuno biolomorfismo, ci si sposta in un insieme che può essere stretto fra due ellissoidi complessi, uno contenuto e uno che lo contiene.}\only<4>{ Per gli ellissoidi complessi, la metrica di Kobayashi può essere calcolata esplicitamente. \qed}
\end{frame}

\begin{frame}[t]
  \frametitle{2. Vicino al bordo, $k_{\Omega}$ e $g$ sono quasi-isometriche}
  \only<1->{
  \begin{thm}
    Sia $\Omega$ un dominio strettamente pseudoconvesso limitato, e sia $k_{\Omega}$ la distanza di Kobayashi su $\Omega$. Allora esiste $C \ge 0$ tale che per ogni $x, y \in \Omega$ vale
    \begin{equation} \label{stimadistanzakobayashi}
      g(x,y)-C \le k_{\Omega}(x,y) \le g(x,y)+C.
    \end{equation}
  \end{thm}
  }
  \only<2-3>{
  \textit{Idea della dimostrazione:} per la maggiorazione, si cercano delle curve che siano quasi-geodetiche, cioè che realizzano la distanza a meno di una costante additiva, e si integra lungo quelle curve.
  }
  \only<3>{

  Per la minorazione, bisogna mostrare che la stima trovata dall'alto è ottimale, cioè che vale la stima dal basso per tutte le curve.
  }
  \only<4>{
  3. Come già osservato, l'invarianza della Gromov-iperbolicità per quasi-isometrie ci permette di ottenere come corollario il teorema di Balogh-Bonk.
  }
\end{frame}

\begin{frame}
  \frametitle{1. Orbite relativamente compatte o iterate compattamente divergenti}
  \only<1-6>{\begin{block}{Teorema (Abate, 1991)}\begin{itshape}
    Sia $X$ una varietà taut e consideriamo $f \in \textnormal{Hol}(X,X)$. Le seguenti affermazioni sono equivalenti:\pause
    \begin{enumerate}
        \item la successione $\{f^k\}_{k \in \mathbb{N}}$ delle iterate di $f$ non è compattamente divergente;\pause
        \item la successione $\{f^k\}_{k \in \mathbb{N}}$ delle iterate di $f$ non contiene alcuna sottosuccessione compattamente divergente;\pause
        \item la successione $\{f^k\}_{k \in \mathbb{N}}$ delle iterate di $f$ è relativamente compatta in $\textnormal{Hol}(X,X)$;\pause
        \item l'orbita di $z$ è relativamente compatta in $X$ per ogni $z \in X$;\pause
        \item esiste $z_0 \in X$ la cui orbita è relativamente compatta in $X$.
    \end{enumerate}
  \end{itshape}\end{block}}
  \only<7-9>{\begin{ex}
    La palla unitaria in $\mathbb{C}^2$ meno l'origine è Kobayashi-iperbolica e $(\lambda,\kappa)$-visibile per ogni $\lambda\ge 1$ e $\kappa\ge 0$, ma non è taut. \setcounter{beamerpauses}{7}\pause La funzione $f(z,w)=(z/2,e^{i\theta}w)$ è un controesempio al teorema di Abate. \pause Dunque, è anche un controesempio al teorema di tipo ``Wolff-Denjoy''.
  \end{ex}}
\end{frame}

\begin{frame}[t]
  \frametitle{2. Convergenza a una costante a meno di sottosuccessioni}
  \only<1-8>{
    \begin{block}{Lemma 1}\begin{itshape}
      Sia $X$ una sottovarietà complessa, connessa e relativamente compatta di una varietà complessa $Y$. \pause Supponiamo che esista $\kappa_0>0$ tale che $X$ sia $(1,\kappa_0)$-visibile. \pause Siano $Z$ una varietà Kobayashi-iperbolica e $\{f_n\}_{n\in\mathbb{N}}\subseteq\textnormal{Hol}(Z,X)$ una successione compattamente divergente. \pause Allora esistono $\xi\in\partial_YX$ e una sottosuccessione $\{f_{n_j}\}_{j\in\mathbb{N}}$ tali che $f_{n_j}(z)\longrightarrow\xi$ per ogni $z\in Z$.
  \end{itshape}\end{block}
  \only<5-8>{\textit{Traccia della dimostrazione:}} \only<5-6>{per assurdo, troviamo (a meno di sottosuccessioni) $z_0,z_1\in Z$ con $k_Z(z_0,z_1)<\kappa_0/2$ e $f_n(z_0)\longrightarrow\xi_0, f_n(z_1)\longrightarrow\xi_1$, dove $\xi_0,\xi_1\in\partial_YX$ e $\xi_0\not=\xi_1$.\setcounter{beamerpauses}{5}\pause

  Le varietà Kobayashi-iperboliche sono connesse da $(1,\kappa)$-simil-geodetiche per $\kappa>0$, quindi prendiamone una $\sigma:[0,T]\longrightarrow Z$ per $\kappa=\kappa_0/2$ con $\sigma(0)=z_0,\sigma(T)=z_1$.}
  \only<7-8>{Si verifica che le curve $f_n\circ\sigma$ sono $(1,\kappa_0)$-simil-geodetiche. \setcounter{beamerpauses}{7}\pause Per visibilità, esiste un compatto $K$ tale che
  $$\emptyset\not=K\cap f_n\big(\sigma([0,T])\big)$$
  per ogni $n$, ma $\sigma([0,T])$ è compatto e $\{f_n\}_{n\in\mathbb{N}}$ è compattamente divergente, contraddizione.\qed}
  }
  \only<9-12>{
    \begin{block}{Lemma 2}\begin{itshape}
      Sia $X$ una sottovarietà Kobayashi-iperbolica e relativamente compatta di una varietà complessa $Y$. \setcounter{beamerpauses}{9}\pause Supponiamo che esista $\kappa_0>0$ tale che $X$ sia $(1,\kappa_0)$-visibile. Sia $F\in\textnormal{Hol}(X,X)$ tale che la successione $\{F^n\}_{n\in\mathbb{N}}$ sia compattamente divergente. \pause Allora esiste $\xi\in\partial_YX$ tale che per ogni funzione $\mu:\mathbb{N}\longrightarrow\mathbb{N}$ strettamente crescente per cui esiste $y_0 \in X$ tale che
      $$\lim_{j\longrightarrow+\infty} k_X\big(F^{\mu(j)}(y_0),y_0\big)=+\infty$$\pause
      si ha
      $$\lim_{j\longrightarrow+\infty} F^{\mu(j)}(z)=\xi$$
      per ogni $z \in X$.
    \end{itshape}\end{block}
  }
  \only<13-24>{
    \textit{Idea della dimostrazione:} \only<13-18>{si costruisce $\nu:\mathbb{N}\longrightarrow\mathbb{N}$ strettamente crescente tale che:\setcounter{beamerpauses}{13}\pause
    \begin{itemize}
      \item si ha $k_X\big(F^{\nu(j)}(x_0),x_0\big) \ge k_X\big(F^k(x_0),x_0\big)$ per ogni $j \in \mathbb{N}$ e per ogni $k \le \nu(j)$;\pause
      \item si ha $\displaystyle\lim_{j\longrightarrow+\infty}k_X\big(F^{\nu(j)}(x_0),x_0\big)=+\infty$;\pause
      \item la successione $\{F^{\nu(j)}(x_0)\}_{j\in\mathbb{N}}$ converge a un certo $\xi\in\partial_YX$.
    \end{itemize}\pause
    Si scelgono $\tau,\tau':\mathbb{N}\longrightarrow\mathbb{N}$ strettamente crescenti tali che $F^{(\mu\circ\tau)(j)}(z)\longrightarrow\xi'\in\partial_YX$ e $\nu\circ\tau'\ge\mu\circ\tau$.\pause
    
    Si applica il seguente fatto.}
    \only<19-24>{siano $\{m_j\}_{j\in\mathbb{N}}$ e $\{m'_j\}_{j\in\mathbb{N}}$ due successioni strettamente crescenti di numeri naturali e $z_0,z'_0\in X$ tali che:\setcounter{beamerpauses}{19}\pause
    \begin{enumerate}
        \item per ogni $j\in\mathbb{N}$ si ha $m_j \ge m'_j$;\pause
        \item per ogni $j\in\mathbb{N}$ e $k \le m_j$ si ha $k_X\big(F^{m_j}(z_0),z_0\big) \ge k_X\big(F^k(z_0),z_0\big)$;\pause
        \item si ha $\displaystyle\lim_{j\longrightarrow+\infty}k_X\big(F^{m_j}(z_0),z_0\big)=\lim_{j\longrightarrow+\infty}k_X\big(F^{m'_j}(z'_0),z_0\big)=+\infty$;\pause
        \item le successioni $\{F^{m_j}(z_0)\}_{j\in\mathbb{N}}$ e $\{F^{m'_j}(z'_0)\}_{j\in\mathbb{N}}$ convergono, rispettivamente, a $\zeta$ e $\zeta'$ in $\partial_YX$;
    \end{enumerate}\pause
    allora $\zeta=\zeta'$.}
  }
  \only<25-30>{
    \begin{block}{Lemma 3}\begin{itshape}
      Sia $X$ una sottovarietà Kobayashi-iperbolica e relativamente compatta di una varietà complessa $Y$. \setcounter{beamerpauses}{25}\pause Supponiamo che esista $\kappa_0>0$ tale che $X$ sia $(1,\kappa_0)$-visibile. Sia $F\in\textnormal{Hol}(X,X)$ tale che la successione $\{F^n\}_{n\in\mathbb{N}}$ sia compattamente divergente.\pause
      
      Supponiamo che esistano un compatto $K\subseteq X$, una funzione strettamente crescente $\mu:\mathbb{N}\longrightarrow\mathbb{N}$ e $\xi\in\partial_YX$ tali che la successione $\{F^{\mu(j)}\}_{j\in\mathbb{N}}$ converge alla costante $\xi$ uniformemente su $K$. \pause Allora la successione $\{F^{\mu(j)}\}_{j\in\mathbb{N}}$ converge alla costante $\xi$ uniformemente sui compatti.
  \end{itshape}\end{block}\pause
  \textit{Idea della dimostrazione:} è un semplice assurdo. Si usano il Lemma 1, la compattezza e il seguente fatto:\pause

  due successioni che convergono a due punti distinti del bordo non possono avere distanza di Kobayashi tendente a $0$.
  }
  \only<31->{
    \begin{block}{Lemma 4}\begin{itshape}
      Sia $X$ una sottovarietà Kobayashi-iperbolica e relativamente compatta di una varietà complessa $Y$. \setcounter{beamerpauses}{31}\pause Supponiamo che esista $\kappa_0>0$ tale che $X$ sia $(1,\kappa_0)$-visibile. Sia $F\in\textnormal{Hol}(X,X)$ tale che la successione $\{F^n\}_{n\in\mathbb{N}}$ sia compattamente divergente.\pause
      
      Per ogni funzione strettamente crescente $\mu:\mathbb{N}\longrightarrow\mathbb{N}$ esistono $\xi\in\partial_YX$ e una sottosuccessione $\{j_n\}_{n\in\mathbb{N}}\subseteq\mathbb{N}$ tali che la successione $\{F^{\mu(j_n)}\}_{n\in\mathbb{N}}$ converge alla costante $\xi$ uniformemente sui compatti.
    \end{itshape}\end{block}\pause
  \textit{Dimostrazione:} Fissiamo $z_0\in X$. \pause Per la compattezza  di $\overline{X}$ e la divergenza dai compatti di $\{F^n\}_{n\in\mathbb{N}}$, esistono $\xi\in\partial_YX$ e una sottosuccessione $\{j_n\}_{n\in\mathbb{N}}\subseteq\mathbb{N}$ tali che $F^{\mu(j_n)}(z_0)\longrightarrow\xi$. \pause Allora la successione $\{F^{\mu(j_n)}\}_{n\in\mathbb{N}}$ converge alla costante $\xi$ uniformemente sul compatto $\{z_0\}$. \pause Si conclude applicando il Lemma 3. \qed
  }
\end{frame}

\begin{frame}[t]
  \frametitle{3. Unicità del limite}
  \only<1->{
    \begin{block}{Teorema}\begin{itshape}
      Sia $X$ una sottovarietà Kobayashi-iperbolica e relativamente compatta di una varietà complessa $Y$. \pause Supponiamo che esista $\kappa_0>0$ tale che $X$ sia $(1,\kappa_0)$-visibile. Sia $F\in\textnormal{Hol}(X,X)$ tale che la successione $\{F^n\}_{n\in\mathbb{N}}$ sia compattamente divergente.\pause
      
      Allora l'insieme delle funzioni limite di $F$ è costituito da un'unica costante.
    \end{itshape}\end{block}
  }
  \only<4->{
    \textit{Idea della dimostrazione:} \only<4-6>{Siano per assurdo $\xi,\eta$ due costanti che siano anche funzioni limite di $F$. \setcounter{beamerpauses}{4}\pause
    
    Caso 1: esiste (e quindi per ogni) $o \in X$ tale che
    $$\limsup_{\nu\longrightarrow+\infty} k_X\big(F^\nu(o),o\big)=+\infty.$$\pause
    Si usano i Lemmi precedenti, in particolare si usa più volte il Lemma 2, per ottenere una contraddizione.}
    \only<7-8>{caso 2: esiste (e quindi per ogni) $o \in X$ tale che
    $$\limsup_{\nu\longrightarrow+\infty} k_X\big(F^\nu(o),o\big)<+\infty.$$\setcounter{beamerpauses}{7}\pause
    Poniamo
    $$G(x_1,x_2):=\lim_{\delta\longrightarrow0}\inf\big\{k_X\big(F^m(x_1),x_2\big)\mid m\in\mathbb{N}, d_Y\big(F^m(x_1),\xi\big)<\delta\big\}$$
    e $\epsilon:=\displaystyle\liminf_{z \longrightarrow\eta}\inf_{y\in K}k_X(z,y)>0.$}
    \only<9->{sia $K$ il compatto dato dalla visibilità per $\xi$ e $\eta$, e prendiamo $q_1,q_2\in K$ tali che
    $$G(q_1,q_2)<\inf_{x_1,x_2}G(x_1,x_2)+\epsilon.$$\setcounter{beamerpauses}{9}\pause
    Usando la visibilità e alcuni dei Lemmi precedenti, troviamo $x^*\in K$ tale che $G(q_1,q_2)\ge G(q_1,x^*)+\epsilon$, contraddizione.}
  }
\end{frame}

\begin{frame}
  \frametitle{Fine}
  \begin{center}
    \LARGE Grazie per l'attenzione!
  \end{center}
\end{frame}

\begin{frame}
  \frametitle{Bibliografia principale}
  \begin{thebibliography}{widest entry}
    \bibitem[A]{A} M. Abate: Iteration theory, compactly divergent sequences and commuting holomorphic maps. \textit{ Ann. Scuola Norm. Sup. Pisa Cl. Sci. Serie IV}, \textbf{18} (1991), no. 2, 167--191
    \bibitem[BM]{BM} G. Bharali, A. Maitra: A weak notion of visibility, a family of examples, and Wolff-Denjoy theorems. \textit{ Ann. Sc. Norm. Super. Pisa Cl. Sci. Serie V}, \textbf{22} (2021), no. 1, 195--240
    \bibitem[BZ1]{BZ1} G. Bharali, A. Zimmer: Goldilocks domains, a weak notion of visibility, and applications. \textit{ Adv. Math.}, \textbf{310} (2017), 377--425
    \bibitem[BZ2]{BZ2} G. Bharali, A. Zimmer: Unbounded visibility domains, the end compactification, and applications. Preprint, arXiv:2206.13869v1 (2022)
    \bibitem[CMS]{CMS} V. S. Chandel, A. Maitra, A. D. Sarkar: Notions of Visibility with respect to the Kobayashi distance: Comparison and Applications. Preprint, arXiv:2111.00549v1 (2021)
  \end{thebibliography}
\end{frame}

\end{document}
